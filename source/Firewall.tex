%
% Firewall
%
% Aleph Objects Firewall
%
% Copyright (C) 2014, 2015, 2016 Aleph Objects, Inc.
%
% This document is licensed under the Creative Commons Attribution 4.0
% International Public License (CC BY-SA 4.0) by Aleph Objects, Inc.
%

\section{Overview}
Aleph Objects has recently deployed pfSense firewalls, replacing OpenBSD.
Most servers and workstations run GNU/Linux, which uses iptables.


\section{iptables}
iptables is part of the Netfilter project and has been included by default in
the Linux kernel for many years.

\begin{figure}[h!]
\includegraphics[keepaspectratio=true,height=1.10\textheight,width=1.00\textwidth,angle=0]{www-netfilter.png}
 \caption{Netfilter Website}
 \label{fig:www-netfilter}
\end{figure}


\section{pfSense}
\href{https://www.pfsense.org/}{pfSense} --- ``Free, open source customized
distribution of FreeBSD specifically tailored for use as a firewall and router
that is entirely managed via web interface.''

pfSense was selected as Aleph Objects core router/firewall for backbone
connections.

\begin{figure}[h!]
\includegraphics[keepaspectratio=true,height=1.10\textheight,width=1.00\textwidth,angle=0]{www-pfsense.png}
 \caption{pfSense Website}
 \label{fig:www-pfsense}
\end{figure}


\subsection{NAT}
Network Address Translation.

\begin{itemize}
 \item VoIP using SIP is often a problem behind a NAT.
 \item Enable Keepalives in Grandstream phones to connect to the Asterisk server.
 \item Disable ALG (Application Level Gateway) in any consumer/home routers.
\end{itemize}


\subsection{Traffic Shaping}
\begin{itemize}
 \item Prioritize admin ssh to firewalls/servers (in case of DoS, etc.)
 \item Prioritize VoIP
 \item De-prioritize SMTP, etc...
\end{itemize}

\subsection{pfBlockerNG}
\begin{itemize}
 \item IP blocklists for botnets, etc.
\end{itemize}


\subsection{Suricata}
Suricata is being used as an Intrusion Detection System.
It is preferred over Snort as Suricata is multithreaded and Snort isn't.

\begin{figure}[h!]
\includegraphics[keepaspectratio=true,height=1.10\textheight,width=1.00\textwidth,angle=0]{www-suricata.png}
 \caption{Suricata Website}
 \label{fig:www-suricata}
\end{figure}

\begin{itemize}
 \item barnyard2
 \item Snort Blacklists
 \item Emerging Threats Blacklists
 \item GeoIP
 \item Alerts, Blocks, Suppress
 \item SID
\end{itemize}


\subsection{DHCP}
For DHCP services, pfSense uses Dnsmasq, which is also used for DNS
forwarding.

\begin{itemize}
 \item Disable IPv6.
 \item tftp netboot installs.
 \item Static mappings.
\end{itemize}


\subsection{NTP}


\subsection{OpenVPN}
\begin{figure}[h!]
\includegraphics[keepaspectratio=true,height=1.10\textheight,width=1.00\textwidth,angle=0]{www-openvpn.png}
 \caption{OpenVPN Website}
 \label{fig:www-openvpn}
\end{figure}


Virtual Private Networks.


\href{https://www.openvpn.net/}{OpenVPN} --- ``OpenVPN is a full-featured open source SSL VPN solution that accommodates a wide range of configurations, including remote access, site-to-site VPNs, Wi-Fi security, and enterprise-scale remote access solutions with load balancing, failover, and fine-grained access-controls.''

\begin{itemize}
 \item Network design (e.g. many point to point, one central server, etc.).
 \item Main OpenVPN server.
 \item Other internal servers.
 \item External servers private connections.
 \item Laptops.
 \item Mobiles.
 \item SSL certificates.
 \item AES-256-CBC is hardware accelerated on pfSense routers.
 \item SHA512 Auth digest algorithm
 \item Hardware Crypto: BSD cryptodev engine
\end{itemize}


pfSense ships with pre-generated DH keys, due to ``heavy computation''.
This can take an hour for 4096.
\begin{minted}[frame=single]{sh}
/usr/bin/openssl dhparam 1024 > /etc/dh-parameters.1024
/usr/bin/openssl dhparam 2048 > /etc/dh-parameters.2048
/usr/bin/openssl dhparam 4096 > /etc/dh-parameters.4096
\end{minted}



\subsection{Captive Portal}
The Captive Portal for Aleph Mountain building wifi services.


\subsection{SSL Certificates}
pfSense makes it very easy to generate Certificate Signing Requests (CSRs),
which can be send to Gandi.net to get issued a ``properly'' signed SSL
certificate.


\subsection{ssh}
OpenSSH from OpenBSD is used. The BSD shell is a bit different from GNU.


\subsection{DNS}
DNS forwarding is provided by Dnsmasq.

\begin{figure}[h!]
\includegraphics[keepaspectratio=true,height=1.10\textheight,width=1.00\textwidth,angle=0]{www-dnsmasq.png}
 \caption{Dnsmasq Website}
 \label{fig:www-dnsmasq}
\end{figure}



\subsection{Routing}
\begin{itemize}
 \item No BGP, OSPF, etc.
 \item Static backbone routes.
 \item WAN failover
\end{itemize}


\subsection{Interfaces}

\begin{itemize}
 \item Gigabit ethernet.
 \item SFP+.
 \item Hardware offloading (e.g. checksums).
\end{itemize}


\subsection{CARP and Synchronization}
CARP can be used to have transparent failover to another firewall, if one
firewall on the network should drop.

Synchronization between CARP firewalls allows easy configuration updates. For
instance, if a configuration change is made to the DHCP server, it can
``instantly'' push to the backup firewall.


\subsection{Reporting}

\begin{figure}[h!]
\includegraphics[keepaspectratio=true,height=1.10\textheight,width=1.00\textwidth,angle=0]{www-ntopng.png}
 \caption{ntopng Website}
 \label{fig:www-ntopng}
\end{figure}

\begin{itemize}
 \item Dashboard.
 \item Darkstat.
 \item ntopng (``Network Top Next Generation'' ?).
 \item S.M.A.R.T.
 \item System Temperatures.
 \item MRTG
 \item RRD
\end{itemize}


\subsection{Install notes}

A few notes from the initial test install:

\begin{itemize}
 \item Released May 18th, 2016.
 \item pfSense-CE-memstick-2.3.1-RELEASE-amd64.img
 \item FreeBSD 10.3 based.
 \item Installer feels like a step back in computing history.
 \item First boot goes to console with lots of useful options.
 \item Web admin wizard mentions pfSense Gold Subscriptions. It doesn't appear to be for non-free software (e.g. isn't baitware).
 \item They sell very nice looking hardware with pfsense pre-installed. With failover systems (CARP).
 \item Load balancing, failover.
 \item Clean and very responsive web interface (based on Bootstrap).
 \item Web based updater to new minor version.
 \item x86 architecture only.
 \item Looks to have good security errata process, following FreeBSD.
 \item Snort threat lists are available. Paid for more recent ones, same as on other snort platforms.
 \item Installation of additional packages is clean, and doesn't appear to offer any non-free.
 \item ClamAV ...
\end{itemize}

