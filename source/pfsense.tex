%
% Firewall
%
% Aleph Objects Firewall
%
% Copyright (C) 2014, 2015, 2016 Aleph Objects, Inc.
%
% This document is licensed under the Creative Commons Attribution 4.0
% International Public License (CC BY-SA 4.0) by Aleph Objects, Inc.
%
\begin{figure}[h!]
\begin{center}
\includegraphics[keepaspectratio=true,height=1.10\textheight,width=1.00\textwidth,angle=0]{www-pfsense.png}
 \caption{pfSense Website}
 \label{fig:www-pfsense}
\end{center}
\end{figure}

\section{Overview}
Aleph Objects has recently deployed pfSense firewalls, replacing OpenBSD.

\href{https://www.pfsense.org/}{pfSense} --- ``Free, open source customized
distribution of FreeBSD specifically tailored for use as a firewall and router
that is entirely managed via web interface.''

pfSense was selected as Aleph Objects core router/firewall for backbone
connections.

\section{Initial Configuration Outline}

These are the the initial configuration steps for a pfSense firewall. Here is an overview of the steps:

\begin{enumerate}
 \item Make serial connection to pfSense firewall, and do basic initial setup.
 \item Connect via ethernet to pfSense firewall with web browser.
 \item Do more initial setup.
 \item Connect pfSense router to Internet.
 \item Update router.
 \item Install new packages.
 \item Configure packages.
 \item Backup \& reboot.
\end{enumerate}


\subsection{Setup pfSense Hardware}
The following pfSense hardware has been tested:

\begin{itemize}
 \item \href{https://store.pfsense.org/SG-2220/}{SG-2220} --- Two 1Gb ethernet ports, dual core 1.7GHz, 2GB RAM, 60GB SSD, single 100-240V power supply, fanless.
 \item \href{https://store.pfsense.org/SG-2440/}{SG-2440} --- Four 1Gb ethernet ports, dual core 1.7GHz, 4GB RAM, 128GB SSD, single 100-240V power supply, fanless.
 \item \href{https://store.pfsense.org/SG-4860/}{SG-4860} --- Six 1Gb ethernet ports, quad core 2.4GHz, 8GB RAM, 128GB SSD, single 100-240V power supply, fanless.
 \item \href{https://store.pfsense.org/XG-2758-HA.aspx}{XG-2758} --- High availability, ten 1Gb ethernet, two 10Gb SFP+ modules, eight core 2.4GHz, 16GB RAM 120GB RAID SSD, single 90~264V power supply, fans, 1U rackmount.
\end{itemize}

Follow these steps to get the hardware set up to configure the pfSense router.

\begin{enumerate}
 \item Leave the power off for now.
 \item Leave the WAN port unplugged for now.
 \item Plug the pfSense provided USB cable into the Console port on the firewall and the USB port of your Debian workstation.
 \item Plug in the LAN ethernet port into the local LAN switch.
\end{enumerate}

\subsection{Setup via Serial Connection}
\begin{figure}[h!]
\begin{center}
\includegraphics[keepaspectratio=true,height=0.50\textheight,width=0.75\textwidth,angle=0]{pfsense-console.png}
 \caption{pfSense Console}
 \label{fig:pfsense-console}
\end{center}
\end{figure}

With the hardware all plugged in, connect with minicom and do some initial configuration via the USB port.
Note, it is possible to start configuration by going to https://192.168.1.1 if you set an IP address on your
Debian workstation to an address in that subnet. But you won't be able to see all the bootup messages and it
will make debugging/controlling the router more difficult. With the serial connection, you can see the bootloader etc.

\begin{enumerate}
 \item Find where the USB device connected, by running \mintinline{sh}{dmesg -T} on your Debian workstation.
Look for a line with USB0, USB1, etc. in it, such as:
\begin{minted}{sh}
usb 1-6: cp210x converter now attached to ttyUSB0
\end{minted}
 \item Run \texttt{minicom} on your Debian workstation to connect to the router, using the USB device from above.
\begin{minted}{sh}
sudo minicom -D /dev/ttyUSB0
\end{minted}

\begin{figure}[h!]
\begin{center}
\includegraphics[keepaspectratio=true,height=0.50\textheight,width=0.75\textwidth,angle=0]{pfsense-minicom-settings.png}
 \caption{pfSense Minicom Settings}
 \label{fig:pfsense-minicom-settings}
\end{center}
\end{figure}
 \item The connection settings are 115200 baud, 8N1, no hardware flow control, no software flow control. Note, hardware flow control is on by default and may prevent you from entering text into the console. To change these settings in minicom, hit \mintinline{sh}{ctrl-a} then the letter \mintinline{sh}{o}. In the menu, with the arrow keys, select \mintinline{sh}{Serial port setup}.
 \item Plug in power to the pfSense router, and watch it boot up in minicom. It takes 1 minute, 20 seconds for a SG-4860 to cold boot with a default configuration.
 \item In the pfSense menu, select \texttt{1) Assign Interfaces}.
%\begin{minted}[fontsize=\footnotesize,highlightcolor=black]{sh}
\begin{minted}[fontsize=\scriptsize,style=bw]{sh}
*** Welcome to pfSense 2.3.2-RELEASE (amd64 full-install) on pfSense ***

 WAN (wan)       -> igb1       -> 
 LAN (lan)       -> igb0       -> v4: 192.168.1.1/24
 OPT1 (opt1)     -> igb2       -> 
 OPT2 (opt2)     -> igb3       -> 
 OPT3 (opt3)     -> igb4       -> 
 OPT4 (opt4)     -> igb5       -> 

 0) Logout (SSH only)                  9) pfTop
 1) Assign Interfaces                 10) Filter Logs
 2) Set interface(s) IP address       11) Restart webConfigurator
 3) Reset webConfigurator password    12) PHP shell + pfSense tools
 4) Reset to factory defaults         13) Update from console
 5) Reboot system                     14) Enable Secure Shell (sshd)
 6) Halt system                       15) Restore recent configuration
 7) Ping host                         16) Restart PHP-FPM
 8) Shell
  
Enter an option: 1

Valid interfaces are:

igb0   00:08:a2:00:00:00   (up) Intel(R) PRO/1000 Network Connection, Version - 
igb1   00:08:a2:00:00:01 (down) Intel(R) PRO/1000 Network Connection, Version - 
igb2   00:08:a2:00:00:02 (down) Intel(R) PRO/1000 Network Connection, Version - 
igb3   00:08:a2:00:00:03 (down) Intel(R) PRO/1000 Network Connection, Version - 
igb4   00:08:a2:00:00:04 (down) Intel(R) PRO/1000 Network Connection, Version - 
igb5   00:08:a2:00:00:05 (down) Intel(R) PRO/1000 Network Connection, Version - 

Do VLANs need to be set up first?
If VLANs will not be used, or only for optional interfaces, it is typical to
say no here and use the webConfigurator to configure VLANs later, if required.

Should VLANs be set up now [y|n]? n

If the names of the interfaces are not known, auto-detection can
be used instead. To use auto-detection, please disconnect all
interfaces before pressing 'a' to begin the process.

Enter the WAN interface name or 'a' for auto-detection 
(igb0 igb1 igb2 igb3 igb4 igb5 or a): igb1

Enter the LAN interface name or 'a' for auto-detection 
NOTE: this enables full Firewalling/NAT mode.
(igb0 igb2 igb3 igb4 igb5 a or nothing if finished): igb0

Optional interface 1 description found: OPT1
Enter the Optional 1 interface name or 'a' for auto-detection
(igb2 igb3 igb4 igb5 a or nothing if finished): 

The interfaces will be assigned as follows:

WAN  -> igb1
LAN  -> igb0

Do you want to proceed [y|n]? y

Writing configuration...done.
One moment while the settings are reloading... done!
\end{minted}

 \item Note the MAC address for the LAN interface.
 \item Copy MAC address to main DHCP/DNS server, and reserve IP address.
 \item Set DHCP for WAN, disable IPv6. This is assuming the WAN is on an interface with DHCP (e.g. cable modem). If it is statically set, set
the address here.
\begin{minted}[fontsize=\scriptsize,style=bw]{sh}
Enter an option: 2

Available interfaces:

1 - WAN (igb1 - dhcp, dhcp6)
2 - LAN (igb0 - static)

Enter the number of the interface you wish to configure: 1

Configure IPv4 address WAN interface via DHCP? (y/n) y

Configure IPv6 address WAN interface via DHCP6? (y/n) n

Enter the new WAN IPv6 address.  Press <ENTER> for none:
> 

Do you want to revert to HTTP as the webConfigurator protocol? (y/n) n

Please wait while the changes are saved to WAN...
 Reloading filter...
 Reloading routing configuration...
 DHCPD...

The IPv4 WAN address has been set to dhcp

Press <ENTER> to continue.
\end{minted}
 \item Set static IP for LAN, using 10.72.9.254/24 as an example. Do not set an IPv6 address. Don't enter a gateway. Don't enable DHCP server (for now, at least).
\begin{minted}[fontsize=\scriptsize,style=bw]{sh}
Enter an option: 2

Available interfaces:

1 - WAN (igb1 - dhcp)
2 - LAN (igb0)

Enter the number of the interface you wish to configure: 2

Enter the new LAN IPv4 address.  Press <ENTER> for none:
> 10.72.9.254

Subnet masks are entered as bit counts (as in CIDR notation) in pfSense.
e.g. 255.255.255.0 = 24
     255.255.0.0   = 16
     255.0.0.0     = 8

Enter the new LAN IPv4 subnet bit count (1 to 31):
> 24

For a WAN, enter the new LAN IPv4 upstream gateway address.
For a LAN, press <ENTER> for none:
> 

Enter the new LAN IPv6 address.  Press <ENTER> for none:
> 

Do you want to enable the DHCP server on LAN? (y/n) n

Do you want to revert to HTTP as the webConfigurator protocol? (y/n) n

Please wait while the changes are saved to LAN...
 Reloading filter...
 Reloading routing configuration...
 DHCPD...

The IPv4 LAN address has been set to 10.72.9.254/24
You can now access the webConfigurator by opening the following URL in your web browser:
                https://10.72.9.254/

Press <ENTER> to continue.
\end{minted}
\end{enumerate}

\subsection{Initial Wizard Setup via Web Browser}

\begin{enumerate}
 \item If needed, set IP address on Debian workstation that is in the same subnet as new firewall. If you have two ethernet interfaces, and the new firewall is plugged into the second ethernet interface, depending on the interface name, run one of:
\begin{minted}{sh}
ifconfig eth0 10.72.9.12
ifconfig eth1 10.72.9.12
ifconfig enp3s0 10.72.9.12
\end{minted}

If you're workstation has only one ethernet connection and the new firewall is plugged into a switch that connects to the workstation, set a second IP address on the primary ethernet interface. It will likely be this:
\begin{minted}{sh}
ifconfig eth0:0 10.72.9.12
\end{minted}

 \item On your Debian workstation, point your browser to the IP address you set on the LAN interface. For example, https://10.72.9.254/
 \item Your browser will issue a warning about the SSL certificate.

\begin{minted}[fontsize=\scriptsize,style=bw]{sh}
Your connection is not private

Attackers might be trying to steal your information from 10.72.9.254 (for example, passwords, messages, or credit cards). NET::ERR_CERT_AUTHORITY_INVALID
\end{minted}

\begin{figure}[h!]
\begin{center}
\includegraphics[keepaspectratio=true,height=0.50\textheight,width=0.75\textwidth,angle=0]{pfsense-invalidssl.png}
 \caption{pfSense Cert Authority Invalid}
 \label{fig:pfsense-invalidssl}
\end{center}
\end{figure}

 \item In your browser hit Advanced, which will display this text:
\begin{minted}[fontsize=\scriptsize,style=bw]{sh}
This server could not prove that it is 10.72.9.254; its security certificate is not trusted by your computer's operating system. This may be caused by a misconfiguration or an attacker intercepting your connection.

Proceed to 10.72.9.254 (unsafe)
\end{minted}

 \item Click Proceed to go through with this operation. A new SSL cert will be set up on the firewall in later steps.

 \item Log in to firewall (e.g. https://10.72.9.254). The initial username is admin and the password is pfsense.

\begin{figure}[h!]
\begin{center}
\includegraphics[keepaspectratio=true,height=0.50\textheight,width=0.75\textwidth,angle=0]{pfsense-login.png}
 \caption{pfSense Login}
 \label{fig:pfsense-login}
\end{center}
\end{figure}

 \item Start Wizard, hit Next.
\begin{figure}[h!]
\begin{center}
\includegraphics[keepaspectratio=true,height=0.50\textheight,width=0.75\textwidth,angle=0]{pfsense-wizard-01.png}
 \caption{pfSense Wizard}
 \label{fig:pfsense-wizard-01}
\end{center}
\end{figure}

 \item pfSense Gold, Next.
\begin{figure}[h!]
\begin{center}
\includegraphics[keepaspectratio=true,height=0.50\textheight,width=0.75\textwidth,angle=0]{pfsense-wizard-02.png}
 \caption{pfSense Gold}
 \label{fig:pfsense-wizard-02}
\end{center}
\end{figure}

 \item Hostname: set hostname and domain. In this example, fw729 and alephobjects.com. DNS servers can be blank for now.
\begin{figure}[h!]
\begin{center}
\includegraphics[keepaspectratio=true,height=0.50\textheight,width=0.75\textwidth,angle=0]{pfsense-wizard-03.png}
 \caption{pfSense Wizard General Information}
 \label{fig:pfsense-wizard-03}
\end{center}
\end{figure}

 \item Time Server Hostname, leave at default. Set timezone to America/Denver.
\begin{figure}[h!]
\begin{center}
\includegraphics[keepaspectratio=true,height=0.50\textheight,width=0.75\textwidth,angle=0]{pfsense-wizard-04.png}
 \caption{pfSense Wizard Timezone}
 \label{fig:pfsense-wizard-04}
\end{center}
\end{figure}

 \item Configure WAN interface. Leave the WAN interface ethernet cable unplugged. In later steps, once we get the basic firewalling etc. running, we can plug in WAN and configure it. For now, take the WAN DHCP defaults.
\begin{figure}[h!]
\begin{center}
\includegraphics[keepaspectratio=true,height=0.50\textheight,width=0.75\textwidth,angle=0]{pfsense-wizard-05.png}
 \caption{pfSense Wizard WAN Configuration}
 \label{fig:pfsense-wizard-05}
\end{center}
\end{figure}

 \item Configure LAN interface. Set the static IP and netmask, in this example 10.72.9.254 and 24.
\begin{figure}[h!]
\begin{center}
\includegraphics[keepaspectratio=true,height=0.50\textheight,width=0.75\textwidth,angle=0]{pfsense-wizard-06.png}
 \caption{pfSense Wizard LAN Configuration}
 \label{fig:pfsense-wizard-06}
\end{center}
\end{figure}

 \item Set Admin WebGUI Password. Choose something unique and following good practices. Use a good password generator.
\begin{figure}[h!]
\begin{center}
\includegraphics[keepaspectratio=true,height=0.50\textheight,width=0.75\textwidth,angle=0]{pfsense-wizard-07.png}
 \caption{pfSense Wizard Admin Password}
 \label{fig:pfsense-wizard-07}
\end{center}
\end{figure}

 \item Hit Reload to use the new configuration.
\begin{figure}[h!]
\begin{center}
\includegraphics[keepaspectratio=true,height=0.50\textheight,width=0.75\textwidth,angle=0]{pfsense-wizard-08.png}
 \caption{pfSense Wizard Reload Configuration}
 \label{fig:pfsense-wizard-08}
\end{center}
\end{figure}

 \item Watch wizard reload. This should take less than a minute.
\begin{figure}[h!]
\begin{center}
\includegraphics[keepaspectratio=true,height=0.50\textheight,width=0.75\textwidth,angle=0]{pfsense-wizard-09.png}
 \caption{pfSense Wizard Reloading}
 \label{fig:pfsense-wizard-09}
\end{center}
\end{figure}

 \item The wizard is now complete. Click \texttt{here} in \texttt{Click here to continue on to pfSense webConfigurator.}
\begin{figure}[h!]
\begin{center}
\includegraphics[keepaspectratio=true,height=0.50\textheight,width=0.75\textwidth,angle=0]{pfsense-wizard-10.png}
 \caption{pfSense Wizard Complete}
 \label{fig:pfsense-wizard-10}
\end{center}
\end{figure}

\end{enumerate}

\subsection{Moar Configuration}
Now that the basic setup of the firewall is done...there is more...
Upon logging in the first time, you are greeted with a basic Dashboard. Menu items are across the top.
\begin{figure}[h!]
\begin{center}
\includegraphics[keepaspectratio=true,height=0.50\textheight,width=0.75\textwidth,angle=0]{pfsense-dashboard-01.png}
 \caption{pfSense Initial Dashboard}
 \label{fig:pfsense-dashboard-01}
\end{center}
\end{figure}


\subsection{Set Up a New User}
%\begin{figure}[h!]
%\begin{center}
%\includegraphics[keepaspectratio=true,height=0.15\textheight,width=0.15\textwidth,angle=0]{pfsense-menu-system-user.png}
% \caption{pfSense Menu, System User Manager}
% \label{fig:pfsense-menu-system-user}
%\end{center}
%\end{figure}

\begin{enumerate}
 \item Go to the top \texttt{System} tab and click \texttt{User Manager}.
\begin{figure}[h!]
\begin{center}
\includegraphics[keepaspectratio=true,height=0.50\textheight,width=0.75\textwidth,angle=0]{pfsense-system-user-users.png}
 \caption{pfSense Menu: System, User Manager, Users}
 \label{fig:pfsense-system-user-users}
\end{center}
\end{figure}
 \item Click Add.

\begin{figure}[h!]
\begin{center}
\includegraphics[keepaspectratio=true,height=0.50\textheight,width=0.75\textwidth,angle=0]{pfsense-system-user-add.png}
 \caption{pfSense Menu: System, User Manager, Add User}
 \label{fig:pfsense-system-user-add}
\end{center}
\end{figure}

 \item Add Username, jebba in this example.
 \item Add a good, well-generated unique password.
 \item Full name.
 \item Experation date leave blank.
 \item Move to add to Group membership ``admins''.
 \item Certificate, leave blank for now (one will be set up later with OpenVPN).
 \item Add ssh key, if you want to ssh in.
 \item Hit Save.
 \item Log out as the admin user by clicking the logout icon in the upper right corner of the pfSense web page.
 \item Log back in as the newly created user in the above steps.

\subsection{Advanced System Configuration}
 \item Goto System --> Advanced. Under Admin Access. Set TCP port to randomish port between 1 and 65535. This will be the new pfSense web interface port address. Max processes: 16 (2 is too low, not sure what is ideal). Check WebGUI redirect to disable port 80. Check enable Secure Shell Server. Disable password login. Pick randomish port for SSH. Check to password protect Console menu. Save.
 \item At this point, you can optionally SSH into the firewall if a key was set up for the user.
 \item Under System --> Advanced, Networking. Uncheck Allow IPv6, to disable IPv6 (yay!). Check boxes for: Hardware Checksum Offloading, Hardware TCP Segmentation Offloading, Hardware Large Receive Offloading. These need to be disabled because/if Suricata is used in Inline mode. If it isn't, these can be unchecked and if the hardware is good/can handle it, it will likely be faster. (Side note, enabling Hardware Checksum Offloading breaks networking in a KVM.) Save.
 \item System --> Advanced, Miscellaneous. Cryptographic Hardware should be set to AES-NI for any hardware from pfSense. For other hardware, check dmesg. Thermal Sensors: Intel Core* CPU on-die thermal sensor. Hard disk standby time, 6 minutes (not sure this really has any effect). Host UUID, check Do NOT send HOST UUID with user agent. Save.
 \item System --> Notifications. Check for Disable Growl Notifications and Disable SMTP. Save.
\end{enumerate}

\subsection{SSL Certificate Manager Setup}
\begin{enumerate}
 \item System --> Cert. Manager. Under Certificates, click Add. Method: Create a Certificate Signing Request. Descriptive Name, use the hostname of the firewall being setup. Key length 4096, Digest Algorithm sha512. Country Code US, etc. Common name, use hostname of firewall. Save.
 \item System --> Cert. Manager, Under Certificates, the new cert added above, click Export Request mini-icon. Gandi: Standard SSL, single address, 3 year. Paste in the CSR exported from the mini-icon into Gandi. Select Apache/ModSSL for Software used in Gandi, and if it says the correct ``Main domain (CN)'', hit Submit in Gandi. Delete any .req file that was downloaded by browser. Gandi: Validation by email (probably). It will take ~10+ minutes to get verification back from Gandi (not instant).
 \item System --> Cert. Manager, Under Certificates. When cert is ready and confirmed at Gandi, hit Get the Certificate. Hit the ``Update CSR'' pencil on the appropriate certificate line. Paste the Gandi cert into Final certificate data. Delete any downloaded copies.
 \item System --> Cert. Manager. Under CAs, click Add. Method: Import an existing Certificate Authority. Descriptive Name: ``GandiStandardSSLCA2''. Get the cert from \url{https://www.gandi.net/static/CAs/GandiStandardSSLCA2.pem} and paste into Certificate data. Certificate Private Key, blank. Save. Note, this has to be done after the above Gandi certificate is added to the firewall.
 \item System --> Cert. Manager, also import any of our own CAs and Certificate Revocations, if any.
\end{enumerate}

\subsection{General Setup}
\begin{enumerate}
 \item System --> General Setup. Check all looks good. Top Navigation: Fixed (Remains visible at top of page). Hostname in Menu: Hostname only. (DNS servers can be bound to particular interfaces here, if needed in multi-WAN). Save.
 \item To use the newly set up certifcate. System --> Advanced, Admin Acess. Change SSL Certificate to the new one. Save. Now go to that new hostname with https and the correct port.
\end{enumerate}

\subsection{Initial Firewall Rules}
\begin{enumerate}
 \item Firewall --> Rules. LAN interface, click the pencil to edit the IPv6 line. Change Action to Reject. Change Source to any. Change Description to ``Default Reject LAN IPv6''. Save. Apply Changes.
 \item Firewall --> Rules. Under LAN interface, click the Copy mini-icon to copy the IPv6 line. Change Action to Block. Interface to WAN. Change Description to ``Default Block WAN IPv6''. Save. Apply Changes.
\end{enumerate}

\subsection{Initial DNS Resolver Setup}
\begin{enumerate}
 \item Services --> DNS Resolver. Enable (default). Network Interfaces: just select LAN and localhost. Outgoing Network Interfaces: WAN, LAN, localhost. Add checks for DHCP Registration, Static DHCP. Save. Apply Changes.
 \item Services --> DNS Resolver, Advanced Settings. Add checks for: Prefetch Support, Prefetch DNS Key Support. Increase Message Cache Size to 50 MB or so (?). Save. Apply Changes.
 \item Services --> Dynamic DNS. Setup, if needed.
\end{enumerate}

\subsection{Initial Logging Setup}
Setup logging to the local firewall. Remote logging will be set up.

\begin{enumerate}
 \item Status --> System Logs, Settings. Add checks to Forward/Reverse Display. GUI Log Entries, increase to 200. Where to show rule descriptions ``Display as second row''. This is where remote logging will be set up... 
 \item Status --> Dashboard. Click the Plus + in the upper right corner. Add the available widgets: Gateways, Thermal Sensors, Traffic Graphs, S.M.A.R.T. Status, Firewall Logs, Interface Statistics, OpenVPN, Services Status, NTP Status.
\end{enumerate}

\subsection{Backup}
Make first backup.

\begin{enumerate}
 \item Diagnostics --> Backup \& Restore. Backup.
 \item Diagnostics --> Reboot and make sure everything comes up clean.
\end{enumerate}

\subsection{Internet Connection}
Make initial connection to the Internet with the new pfSense firewall.


At this point, this presumes the WAN interface isn't up and routing actual Internet traffic. It is better to get the router as configured as possible before actually using the WAN interface. Assuming the firewall is on the LAN and being configured, it can use the gateway that is on its LAN interface. When configuration is finalized and the router is deployed, the WAN interface will carry Internet traffic. To to this, add a route by: Interfaces --> LAN. Under IPv4 Upstream gateway, click Add a new gateway. Add the LAN gateway info, and check is as Default gateway (3000). Save. Apply Changes..

\begin{enumerate}
 \item System --> Routing. Click to edit the mini pencil icon on the Gateway line listed as Default. Monitor IP: something appropriate upstream, can use 8.8.8.8. Note: on high latency connections such as satellite, hit Display Advanced and increase Latency thresholds (750, 2500), Packet Loss thresholds (15, 25), Probe Interval (1000), Loss Interval (3000). Save. Apply Changes.
\end{enumerate}

\subsection{Update \& Install Packages}
\begin{enumerate}
 \item System --> Update, System Update. Check that the Status is ``Up to date.'' If it needs updating, update.
 \item System --> Package Manager, Installed Packages. The first time here, you need to click on Available Packages (presumably to download latest package header info). Then go back to Installed Packages. If there are any Installed Packages that have a Newer version available, click the mini icon to update the package. Confirm.
 \item System --> Package Manager, Available Packages. Install: Cron, ntopng, openvpn-client-export, pfBlockerNG, RRD\_Summary, Status\_Traffic\_Totals, sudo, suricata.
 \item System --> sudo. Add user, optionally.
\end{enumerate}


\subsection{OpenVPN}
OpenVPN is run by a collection of pfSense firewalls.

\begin{enumerate}
 \item OpenVPN setup.
 \item System --> Cert. Manager. Set up internal certificate authority.
 \item System --> Cert. Manager. Create internal server certificate.
 \item SSH into firewall. pfSense ships with pre-generated DH keys, due to ``heavy computation''. This can take an hour for 4096.
\begin{minted}{sh}
/usr/bin/openssl dhparam 1024 > /etc/dh-parameters.1024
/usr/bin/openssl dhparam 2048 > /etc/dh-parameters.2048
/usr/bin/openssl dhparam 4096 > /etc/dh-parameters.4096
\end{minted}
 \item VPN --> OpenVPN. Set up VPN server. Server Mode: Remote Access (SSL/TLS + User Auth). Backend for Authentication: Local Database (will be FreeRADIUS at some point). Protocol: UDP. Local Port: something randomish. Peer Certificate Authority: use the interal CA created earlier. Server Certificate: Use the server certificate created earlier. DH Parameter length (bits): 4096. Encryption Algorithm: AES-256-CBC (256-bit), this has hardware crypto support on pfSense routers. Auth digest algorithm: SHA512 (512-bit). Hardware Crypto: BSD Cryptodev engine- RSA, DSA, DH, AES-128-CBC, AES-192-CBC, AES-256-CBC. Certificate Depth: One. Strict User-CN Matching: check Enforce Match once it is confirmed working. IPv4 Tunnel Network: Set the new VPN network. IPv6 Tunnel Network: leave blank. Redirect Gateway: unchecked.  IPv4 Local network(s): set the local LAN network. IPv6 Local network(s): leave blank. Compression: Enabled with Adaptive Compression. Inter-client communication: Checked. Disable IPv6: Checked. Dynamic IP: Unchecked, at least for now. Topology: subnet. DNS Default Domain: Checked, and set domain. DNS Server enable: Checked. DNS Server 1: enter servers. Save.
 \item System --> User Manager. Add user for VPN. Certificate: Check to create user certificate. Descriptive name: username.domainname. Certificate Authority: Select internal CA created above. Key length: 4096. Lifetime: 1095.
 \item VPN --> OpenVPN, Client Export. Remote Access Server: Select the VPN server created earlier. Verify Server CN: automatic. Block Outside DNS: Checked. At the bottom, export as Standard Configurations, Archive to use with another pfSense server. To use with OpenVPN in F-Droid (Android), use Inline Configurations (Android).
\end{enumerate}


\subsection{Turn off Internet via LAN}
\begin{enumerate}
 \item Interfaces --> LAN. When you're done using the LAN as any sort of gateway. Change IPv4 Upstream gateway to None. Save. Apply Changes.
\end{enumerate}

\subsection{More Backup}
Make another backup.

\begin{enumerate}
 \item Diagnostics --> Backup \& Restore. Backup.
 \item Diagnostics --> Reboot and make sure everything comes up clean.
\end{enumerate}


\section{NAT}
Network Address Translation.

\begin{itemize}
 \item VoIP using SIP is often a problem behind a NAT.
 \item Enable Keepalives in Grandstream phones to connect to the Asterisk server.
 \item Disable ALG (Application Level Gateway) in any consumer/home routers.
\end{itemize}


\section{Traffic Shaping}
\begin{itemize}
 \item Prioritize admin ssh to firewalls/servers (in case of DoS, etc.)
 \item Prioritize VoIP
 \item De-prioritize SMTP, etc...
\end{itemize}

\section{pfBlockerNG}
\begin{itemize}
 \item IP blocklists for botnets, etc.
\end{itemize}


\section{Suricata}
Suricata is being used as an Intrusion Detection System.
It is preferred over Snort as Suricata is multithreaded and Snort isn't.

\begin{figure}[h!]
\begin{center}
\includegraphics[keepaspectratio=true,height=0.50\textheight,width=0.75\textwidth,angle=0]{www-suricata.png}
 \caption{Suricata Website}
 \label{fig:www-suricata}
\end{center}
\end{figure}

\begin{itemize}
 \item barnyard2
 \item Snort Blacklists
 \item Emerging Threats Blacklists
 \item GeoIP
 \item Alerts, Blocks, Suppress
 \item SID
\end{itemize}


\section{DHCP}
For DHCP services, pfSense uses Dnsmasq, which is also used for DNS
forwarding.


%Services --> DHCP Server. Note: for netboot install of a workstation. Set up DHCP address. Network Booting, show advanced. Enable network booting for that host. Set Next Server to the IP address of the tftp server. Set XXX to ``jessie_crypto'' for encrypted install or ``jessie'' for a regular install.
\begin{enumerate}
 \item Services --> DHCP Server. Network Booting, click Display Advanced. Check box for Enables Network Booting. Set Default BIOS file name to jessie\_crypto/pxelinux.0 or jessie/pxelinux.0. Set Next Server to IP address of tftp server. Save.
 \item Services --> DHCP Server. Go to the bottom and hit Add. Add the MAC address, Client Identifier (hostname), IP Address, Hostname, Netboot Filename (we probably don't need it in the general config), and TFTP Server.
\end{enumerate}

\begin{itemize}
 \item Disable IPv6.
 \item tftp netboot installs.
 \item Static mappings.
\end{itemize}


\section{NTP}


\section{OpenVPN}
\begin{figure}[h!]
\begin{center}
\includegraphics[keepaspectratio=true,height=0.50\textheight,width=0.75\textwidth,angle=0]{www-openvpn.png}
 \caption{OpenVPN Website}
 \label{fig:www-openvpn}
\end{center}
\end{figure}


Virtual Private Networks.


\href{https://www.openvpn.net/}{OpenVPN} --- ``OpenVPN is a full-featured open source SSL VPN solution that accommodates a wide range of configurations, including remote access, site-to-site VPNs, Wi-Fi security, and enterprise-scale remote access solutions with load balancing, failover, and fine-grained access-controls.''

\begin{itemize}
 \item Network design (e.g. many point to point, one central server, etc.).
 \item Main OpenVPN server.
 \item Other internal servers.
 \item External servers private connections.
 \item Laptops.
 \item Mobiles.
 \item SSL certificates.
 \item AES-256-CBC is hardware accelerated on pfSense routers.
 \item SHA512 Auth digest algorithm
 \item Hardware Crypto: BSD cryptodev engine
\end{itemize}


pfSense ships with pre-generated DH keys, due to ``heavy computation''.
This can take an hour for 4096.
\begin{minted}{sh}
/usr/bin/openssl dhparam 1024 > /etc/dh-parameters.1024
/usr/bin/openssl dhparam 2048 > /etc/dh-parameters.2048
/usr/bin/openssl dhparam 4096 > /etc/dh-parameters.4096
\end{minted}



\section{Captive Portal}
The Captive Portal for Aleph Mountain building wifi services.


\section{SSL Certificates}
pfSense makes it very easy to generate Certificate Signing Requests (CSRs),
which can be send to Gandi.net to get issued a ``properly'' signed SSL
certificate.


\section{ssh}
OpenSSH from OpenBSD is used. The BSD shell is a bit different from GNU.


\section{DNS}
DNS forwarding is provided by Dnsmasq.

\begin{figure}[h!]
\begin{center}
\includegraphics[keepaspectratio=true,height=0.50\textheight,width=0.75\textwidth,angle=0]{www-dnsmasq.png}
 \caption{Dnsmasq Website}
 \label{fig:www-dnsmasq}
\end{center}
\end{figure}



\section{Routing}
\begin{itemize}
 \item No BGP, OSPF, etc.
 \item Static backbone routes.
 \item WAN failover
\end{itemize}


\section{Interfaces}

\begin{itemize}
 \item Gigabit ethernet.
 \item SFP+.
 \item Hardware offloading (e.g. checksums).
\end{itemize}


\section{CARP and Synchronization}
CARP can be used to have transparent failover to another firewall, if one
firewall on the network should drop.

Synchronization between CARP firewalls allows easy configuration updates. For
instance, if a configuration change is made to the DHCP server, it can
``instantly'' push to the backup firewall.


\section{Reporting}

\begin{figure}[h!]
\begin{center}
\includegraphics[keepaspectratio=true,height=0.50\textheight,width=0.75\textwidth,angle=0]{www-ntopng.png}
 \caption{ntopng Website}
 \label{fig:www-ntopng}
\end{center}
\end{figure}

\begin{itemize}
 \item Dashboard.
 \item Darkstat.
 \item ntopng (``Network Top Next Generation'' ?).
 \item S.M.A.R.T.
 \item System Temperatures.
 \item MRTG
 \item RRD
\end{itemize}


